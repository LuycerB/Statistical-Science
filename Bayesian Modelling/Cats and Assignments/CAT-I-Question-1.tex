% Options for packages loaded elsewhere
\PassOptionsToPackage{unicode}{hyperref}
\PassOptionsToPackage{hyphens}{url}
%
\documentclass[
]{article}
\usepackage{lmodern}
\usepackage{amssymb,amsmath}
\usepackage{ifxetex,ifluatex}
\ifnum 0\ifxetex 1\fi\ifluatex 1\fi=0 % if pdftex
  \usepackage[T1]{fontenc}
  \usepackage[utf8]{inputenc}
  \usepackage{textcomp} % provide euro and other symbols
\else % if luatex or xetex
  \usepackage{unicode-math}
  \defaultfontfeatures{Scale=MatchLowercase}
  \defaultfontfeatures[\rmfamily]{Ligatures=TeX,Scale=1}
\fi
% Use upquote if available, for straight quotes in verbatim environments
\IfFileExists{upquote.sty}{\usepackage{upquote}}{}
\IfFileExists{microtype.sty}{% use microtype if available
  \usepackage[]{microtype}
  \UseMicrotypeSet[protrusion]{basicmath} % disable protrusion for tt fonts
}{}
\makeatletter
\@ifundefined{KOMAClassName}{% if non-KOMA class
  \IfFileExists{parskip.sty}{%
    \usepackage{parskip}
  }{% else
    \setlength{\parindent}{0pt}
    \setlength{\parskip}{6pt plus 2pt minus 1pt}}
}{% if KOMA class
  \KOMAoptions{parskip=half}}
\makeatother
\usepackage{xcolor}
\IfFileExists{xurl.sty}{\usepackage{xurl}}{} % add URL line breaks if available
\IfFileExists{bookmark.sty}{\usepackage{bookmark}}{\usepackage{hyperref}}
\hypersetup{
  pdftitle={CAT I Question I},
  pdfauthor={124384 Luycer Bosire},
  hidelinks,
  pdfcreator={LaTeX via pandoc}}
\urlstyle{same} % disable monospaced font for URLs
\usepackage[margin=1in]{geometry}
\usepackage{color}
\usepackage{fancyvrb}
\newcommand{\VerbBar}{|}
\newcommand{\VERB}{\Verb[commandchars=\\\{\}]}
\DefineVerbatimEnvironment{Highlighting}{Verbatim}{commandchars=\\\{\}}
% Add ',fontsize=\small' for more characters per line
\usepackage{framed}
\definecolor{shadecolor}{RGB}{248,248,248}
\newenvironment{Shaded}{\begin{snugshade}}{\end{snugshade}}
\newcommand{\AlertTok}[1]{\textcolor[rgb]{0.94,0.16,0.16}{#1}}
\newcommand{\AnnotationTok}[1]{\textcolor[rgb]{0.56,0.35,0.01}{\textbf{\textit{#1}}}}
\newcommand{\AttributeTok}[1]{\textcolor[rgb]{0.77,0.63,0.00}{#1}}
\newcommand{\BaseNTok}[1]{\textcolor[rgb]{0.00,0.00,0.81}{#1}}
\newcommand{\BuiltInTok}[1]{#1}
\newcommand{\CharTok}[1]{\textcolor[rgb]{0.31,0.60,0.02}{#1}}
\newcommand{\CommentTok}[1]{\textcolor[rgb]{0.56,0.35,0.01}{\textit{#1}}}
\newcommand{\CommentVarTok}[1]{\textcolor[rgb]{0.56,0.35,0.01}{\textbf{\textit{#1}}}}
\newcommand{\ConstantTok}[1]{\textcolor[rgb]{0.00,0.00,0.00}{#1}}
\newcommand{\ControlFlowTok}[1]{\textcolor[rgb]{0.13,0.29,0.53}{\textbf{#1}}}
\newcommand{\DataTypeTok}[1]{\textcolor[rgb]{0.13,0.29,0.53}{#1}}
\newcommand{\DecValTok}[1]{\textcolor[rgb]{0.00,0.00,0.81}{#1}}
\newcommand{\DocumentationTok}[1]{\textcolor[rgb]{0.56,0.35,0.01}{\textbf{\textit{#1}}}}
\newcommand{\ErrorTok}[1]{\textcolor[rgb]{0.64,0.00,0.00}{\textbf{#1}}}
\newcommand{\ExtensionTok}[1]{#1}
\newcommand{\FloatTok}[1]{\textcolor[rgb]{0.00,0.00,0.81}{#1}}
\newcommand{\FunctionTok}[1]{\textcolor[rgb]{0.00,0.00,0.00}{#1}}
\newcommand{\ImportTok}[1]{#1}
\newcommand{\InformationTok}[1]{\textcolor[rgb]{0.56,0.35,0.01}{\textbf{\textit{#1}}}}
\newcommand{\KeywordTok}[1]{\textcolor[rgb]{0.13,0.29,0.53}{\textbf{#1}}}
\newcommand{\NormalTok}[1]{#1}
\newcommand{\OperatorTok}[1]{\textcolor[rgb]{0.81,0.36,0.00}{\textbf{#1}}}
\newcommand{\OtherTok}[1]{\textcolor[rgb]{0.56,0.35,0.01}{#1}}
\newcommand{\PreprocessorTok}[1]{\textcolor[rgb]{0.56,0.35,0.01}{\textit{#1}}}
\newcommand{\RegionMarkerTok}[1]{#1}
\newcommand{\SpecialCharTok}[1]{\textcolor[rgb]{0.00,0.00,0.00}{#1}}
\newcommand{\SpecialStringTok}[1]{\textcolor[rgb]{0.31,0.60,0.02}{#1}}
\newcommand{\StringTok}[1]{\textcolor[rgb]{0.31,0.60,0.02}{#1}}
\newcommand{\VariableTok}[1]{\textcolor[rgb]{0.00,0.00,0.00}{#1}}
\newcommand{\VerbatimStringTok}[1]{\textcolor[rgb]{0.31,0.60,0.02}{#1}}
\newcommand{\WarningTok}[1]{\textcolor[rgb]{0.56,0.35,0.01}{\textbf{\textit{#1}}}}
\usepackage{graphicx,grffile}
\makeatletter
\def\maxwidth{\ifdim\Gin@nat@width>\linewidth\linewidth\else\Gin@nat@width\fi}
\def\maxheight{\ifdim\Gin@nat@height>\textheight\textheight\else\Gin@nat@height\fi}
\makeatother
% Scale images if necessary, so that they will not overflow the page
% margins by default, and it is still possible to overwrite the defaults
% using explicit options in \includegraphics[width, height, ...]{}
\setkeys{Gin}{width=\maxwidth,height=\maxheight,keepaspectratio}
% Set default figure placement to htbp
\makeatletter
\def\fps@figure{htbp}
\makeatother
\setlength{\emergencystretch}{3em} % prevent overfull lines
\providecommand{\tightlist}{%
  \setlength{\itemsep}{0pt}\setlength{\parskip}{0pt}}
\setcounter{secnumdepth}{-\maxdimen} % remove section numbering

\title{CAT I Question I}
\author{124384 Luycer Bosire}
\date{2/7/2021}

\begin{document}
\maketitle

\textbf{Question 1: Single parameter poisson}

Consider the count of airline crashes per year over a 10-year period
form 1976 to 1985:

\textbf{\emph{a. Show that the Gamma distribution is the conjugate prior
for a Poisson mean.}}

Solution in the steps below;

\begin{enumerate}
\def\labelenumi{\arabic{enumi}.}
\tightlist
\item
  write down the likelihood \(p(X|\mu)\) for \(\mu\).
\end{enumerate}

The pdf and likelihood function of Poisson Distribution with parameter
\(\mu\) is given by;

\[ f(x, \mu) = \frac {e^{-\mu}\mu^{x}}{x!} \]

\[ L(\mu; \pmb x) = \frac {e^{-n\mu}\theta^{n\bar x}}{\prod_{i=1}^nx_i!}\propto e^{-n\mu}\mu^{n\bar x} \]

\begin{enumerate}
\def\labelenumi{\arabic{enumi}.}
\setcounter{enumi}{1}
\tightlist
\item
  Write down the prior density for \(\mu\).
\end{enumerate}

The prior density Gamma with parameter \[n, \lambda\] is given by;

\[ f(n, \lambda)= \frac {\beta^\alpha} {\Gamma(\alpha)} {\lambda^{(\alpha-1)}} {e^ {(-\lambda\beta)}} \]
3. Multiply them together to obtain the posterior density, and notice
that it has the same form as the gamma distribution.

Using Bayes Theorem which we can write in words as
\[\text{posterior} \propto \text{likelihood} \times \text{prior},\]

In this case, the likelihood \(p(D | \mu)\) is given by
\[p(D | \mu) \propto e^{-n\mu}\mu^{n\bar x}\]

\begin{Shaded}
\begin{Highlighting}[]
\NormalTok{    y <-}\StringTok{ }\KeywordTok{c}\NormalTok{(}\DecValTok{24}\NormalTok{,}\DecValTok{25}\NormalTok{,}\DecValTok{31}\NormalTok{,}\DecValTok{31}\NormalTok{,}\DecValTok{22}\NormalTok{,}\DecValTok{21}\NormalTok{,}\DecValTok{26}\NormalTok{,}\DecValTok{20}\NormalTok{,}\DecValTok{16}\NormalTok{,}\DecValTok{22}\NormalTok{)}
\NormalTok{    n<-}\StringTok{ }\KeywordTok{length}\NormalTok{(y)}
\NormalTok{    n}
\end{Highlighting}
\end{Shaded}

\begin{verbatim}
## [1] 10
\end{verbatim}

\begin{Shaded}
\begin{Highlighting}[]
\NormalTok{    bar_x<-}\StringTok{ }\KeywordTok{mean}\NormalTok{(y)}
\NormalTok{    bar_x}
\end{Highlighting}
\end{Shaded}

\begin{verbatim}
## [1] 23.8
\end{verbatim}

where n = 10 and mean = 23.8 so our likelihood is
\[p(D | \mu) \propto e^{-10\mu}\mu^{10*23.8}\]

Combining the prior and the likelihood we get;

\[ p(D | \mu) \propto (e^{-10\mu}\mu^{10*23.8}) ({\frac {\beta^\alpha} {\Gamma(\alpha)} {\lambda^{(\alpha-1)}} {e^ {(-\lambda\beta)}}}) \propto \lambda ^{(n+\alpha-1)} e^{-(n+\beta)\lambda} \]
At this point we again apply the ``trick'' of recognizing this density
as the density of a Gamma distribution - specifically, the Gamma
distribution with parameters

\[(\alpha+n \bar x,\beta+n) \propto (\alpha+10*23.8,\beta+10)\].

\end{document}
