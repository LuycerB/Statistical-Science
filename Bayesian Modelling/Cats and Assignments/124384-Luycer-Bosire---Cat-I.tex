% Options for packages loaded elsewhere
\PassOptionsToPackage{unicode}{hyperref}
\PassOptionsToPackage{hyphens}{url}
%
\documentclass[
]{article}
\usepackage{lmodern}
\usepackage{amssymb,amsmath}
\usepackage{ifxetex,ifluatex}
\ifnum 0\ifxetex 1\fi\ifluatex 1\fi=0 % if pdftex
  \usepackage[T1]{fontenc}
  \usepackage[utf8]{inputenc}
  \usepackage{textcomp} % provide euro and other symbols
\else % if luatex or xetex
  \usepackage{unicode-math}
  \defaultfontfeatures{Scale=MatchLowercase}
  \defaultfontfeatures[\rmfamily]{Ligatures=TeX,Scale=1}
\fi
% Use upquote if available, for straight quotes in verbatim environments
\IfFileExists{upquote.sty}{\usepackage{upquote}}{}
\IfFileExists{microtype.sty}{% use microtype if available
  \usepackage[]{microtype}
  \UseMicrotypeSet[protrusion]{basicmath} % disable protrusion for tt fonts
}{}
\makeatletter
\@ifundefined{KOMAClassName}{% if non-KOMA class
  \IfFileExists{parskip.sty}{%
    \usepackage{parskip}
  }{% else
    \setlength{\parindent}{0pt}
    \setlength{\parskip}{6pt plus 2pt minus 1pt}}
}{% if KOMA class
  \KOMAoptions{parskip=half}}
\makeatother
\usepackage{xcolor}
\IfFileExists{xurl.sty}{\usepackage{xurl}}{} % add URL line breaks if available
\IfFileExists{bookmark.sty}{\usepackage{bookmark}}{\usepackage{hyperref}}
\hypersetup{
  pdftitle={Bayesian Cat I},
  pdfauthor={124384 Luycer Bosire},
  hidelinks,
  pdfcreator={LaTeX via pandoc}}
\urlstyle{same} % disable monospaced font for URLs
\usepackage[margin=1in]{geometry}
\usepackage{color}
\usepackage{fancyvrb}
\newcommand{\VerbBar}{|}
\newcommand{\VERB}{\Verb[commandchars=\\\{\}]}
\DefineVerbatimEnvironment{Highlighting}{Verbatim}{commandchars=\\\{\}}
% Add ',fontsize=\small' for more characters per line
\usepackage{framed}
\definecolor{shadecolor}{RGB}{248,248,248}
\newenvironment{Shaded}{\begin{snugshade}}{\end{snugshade}}
\newcommand{\AlertTok}[1]{\textcolor[rgb]{0.94,0.16,0.16}{#1}}
\newcommand{\AnnotationTok}[1]{\textcolor[rgb]{0.56,0.35,0.01}{\textbf{\textit{#1}}}}
\newcommand{\AttributeTok}[1]{\textcolor[rgb]{0.77,0.63,0.00}{#1}}
\newcommand{\BaseNTok}[1]{\textcolor[rgb]{0.00,0.00,0.81}{#1}}
\newcommand{\BuiltInTok}[1]{#1}
\newcommand{\CharTok}[1]{\textcolor[rgb]{0.31,0.60,0.02}{#1}}
\newcommand{\CommentTok}[1]{\textcolor[rgb]{0.56,0.35,0.01}{\textit{#1}}}
\newcommand{\CommentVarTok}[1]{\textcolor[rgb]{0.56,0.35,0.01}{\textbf{\textit{#1}}}}
\newcommand{\ConstantTok}[1]{\textcolor[rgb]{0.00,0.00,0.00}{#1}}
\newcommand{\ControlFlowTok}[1]{\textcolor[rgb]{0.13,0.29,0.53}{\textbf{#1}}}
\newcommand{\DataTypeTok}[1]{\textcolor[rgb]{0.13,0.29,0.53}{#1}}
\newcommand{\DecValTok}[1]{\textcolor[rgb]{0.00,0.00,0.81}{#1}}
\newcommand{\DocumentationTok}[1]{\textcolor[rgb]{0.56,0.35,0.01}{\textbf{\textit{#1}}}}
\newcommand{\ErrorTok}[1]{\textcolor[rgb]{0.64,0.00,0.00}{\textbf{#1}}}
\newcommand{\ExtensionTok}[1]{#1}
\newcommand{\FloatTok}[1]{\textcolor[rgb]{0.00,0.00,0.81}{#1}}
\newcommand{\FunctionTok}[1]{\textcolor[rgb]{0.00,0.00,0.00}{#1}}
\newcommand{\ImportTok}[1]{#1}
\newcommand{\InformationTok}[1]{\textcolor[rgb]{0.56,0.35,0.01}{\textbf{\textit{#1}}}}
\newcommand{\KeywordTok}[1]{\textcolor[rgb]{0.13,0.29,0.53}{\textbf{#1}}}
\newcommand{\NormalTok}[1]{#1}
\newcommand{\OperatorTok}[1]{\textcolor[rgb]{0.81,0.36,0.00}{\textbf{#1}}}
\newcommand{\OtherTok}[1]{\textcolor[rgb]{0.56,0.35,0.01}{#1}}
\newcommand{\PreprocessorTok}[1]{\textcolor[rgb]{0.56,0.35,0.01}{\textit{#1}}}
\newcommand{\RegionMarkerTok}[1]{#1}
\newcommand{\SpecialCharTok}[1]{\textcolor[rgb]{0.00,0.00,0.00}{#1}}
\newcommand{\SpecialStringTok}[1]{\textcolor[rgb]{0.31,0.60,0.02}{#1}}
\newcommand{\StringTok}[1]{\textcolor[rgb]{0.31,0.60,0.02}{#1}}
\newcommand{\VariableTok}[1]{\textcolor[rgb]{0.00,0.00,0.00}{#1}}
\newcommand{\VerbatimStringTok}[1]{\textcolor[rgb]{0.31,0.60,0.02}{#1}}
\newcommand{\WarningTok}[1]{\textcolor[rgb]{0.56,0.35,0.01}{\textbf{\textit{#1}}}}
\usepackage{graphicx,grffile}
\makeatletter
\def\maxwidth{\ifdim\Gin@nat@width>\linewidth\linewidth\else\Gin@nat@width\fi}
\def\maxheight{\ifdim\Gin@nat@height>\textheight\textheight\else\Gin@nat@height\fi}
\makeatother
% Scale images if necessary, so that they will not overflow the page
% margins by default, and it is still possible to overwrite the defaults
% using explicit options in \includegraphics[width, height, ...]{}
\setkeys{Gin}{width=\maxwidth,height=\maxheight,keepaspectratio}
% Set default figure placement to htbp
\makeatletter
\def\fps@figure{htbp}
\makeatother
\setlength{\emergencystretch}{3em} % prevent overfull lines
\providecommand{\tightlist}{%
  \setlength{\itemsep}{0pt}\setlength{\parskip}{0pt}}
\setcounter{secnumdepth}{-\maxdimen} % remove section numbering

\title{Bayesian Cat I}
\author{124384 Luycer Bosire}
\date{2/6/2021}

\begin{document}
\maketitle

\textbf{Question 1: Single parameter poisson}

Consider the count of airline crashes per year over a 10-year period
form 1976 to 1985:

\begin{verbatim}
y <- c(24,25,31,31,22,21,26,20,16,22)
n<- length(y)
n
bar_x<- mean(y)
bar_x
\end{verbatim}

\textbf{\emph{a. Show that the Gamma distribution is the conjugate prior
for a Poisson mean.}}

Solution in the steps below;

\begin{enumerate}
\def\labelenumi{\arabic{enumi}.}
\tightlist
\item
  write down the likelihood \(p(X|\lambda)\) for \(\lambda\).
\end{enumerate}

The pdf and likelihood function of Poisson Distribution with parameter
\(\lambda\) is given by;

\[ P(x| \lambda) = \frac {e^{-\lambda}\lambda^{x}}{x!} \]

\[ L(\lambda; \pmb x) = \frac {e^{-n\lambda}\lambda^{n\bar x}}{\prod_{i=1}^nx_i!}\propto e^{-n\lambda}\lambda^{n\bar x} \]

\begin{enumerate}
\def\labelenumi{\arabic{enumi}.}
\setcounter{enumi}{1}
\tightlist
\item
  Write down the prior density for \(\lambda\).
\end{enumerate}

The prior density Gamma with parameter \[n, \lambda\] is given by;

\[ P(x |\lambda)= \frac {\beta^\alpha} {\Gamma(\alpha)} {\lambda^{(\alpha-1)}} {e^ {(-\lambda\beta)}} \]

Ignoring the constant term:
\[P(\lambda)=\lambda^{\alpha-1}e^{-\beta\lambda}\] 3. Multiply them
together to obtain the posterior density, and notice that it has the
same form as the gamma distribution.

Using Bayes Rule

\[\text{posterior} \propto \text{likelihood} \times \text{prior},\] That
is \[P(\lambda|x)=P(x|\lambda) \times P(\lambda)\]

Therefore:
\[P(\lambda|x)=e^{-n\lambda}\lambda^{n\bar x}\times\lambda^{\alpha-1}e^{-\beta\lambda}\]
\[P(\lambda|x)=\lambda^{n\bar x+\alpha-1}e^{-\lambda(n+\beta)}\] Thus we
can notice that the posterior takes a Gamma form:
\[Post(\lambda|x)∼Gamma(\alpha^*,\beta^*)\] Where
\[\alpha^*=n\bar x+\alpha \] \[\beta^*=n+\beta\]

\textbf{\emph{b. Show graphics of the resulting posterior distribution
for choices of various gamma priors}}

Confirming the code for different values of alpha and beta:

\begin{Shaded}
\begin{Highlighting}[]
\KeywordTok{library}\NormalTok{(shiny)}
\KeywordTok{library}\NormalTok{(tidyverse)}
\end{Highlighting}
\end{Shaded}

\begin{verbatim}
## -- Attaching packages --------------------------------------- tidyverse 1.3.0 --
\end{verbatim}

\begin{verbatim}
## v ggplot2 3.3.3     v purrr   0.3.4
## v tibble  3.0.5     v dplyr   1.0.3
## v tidyr   1.1.2     v stringr 1.4.0
## v readr   1.4.0     v forcats 0.5.1
\end{verbatim}

\begin{verbatim}
## -- Conflicts ------------------------------------------ tidyverse_conflicts() --
## x dplyr::filter() masks stats::filter()
## x dplyr::lag()    masks stats::lag()
\end{verbatim}

\begin{Shaded}
\begin{Highlighting}[]
\CommentTok{# ui codes ----------------------------------------------------------------}

\NormalTok{ui <-}\StringTok{ }\KeywordTok{fluidPage}\NormalTok{(}
  \KeywordTok{sidebarLayout}\NormalTok{(}
    \KeywordTok{sidebarPanel}\NormalTok{(}
    \KeywordTok{sliderInput}\NormalTok{(}\StringTok{"n"}\NormalTok{, }\StringTok{"Play to increase sample size"}\NormalTok{, }\DataTypeTok{min =} \DecValTok{1}\NormalTok{,}
                \DataTypeTok{max =} \DecValTok{10}\NormalTok{, }\DataTypeTok{step =} \DecValTok{1}\NormalTok{,}\DataTypeTok{value =} \DecValTok{10}\NormalTok{, }\DataTypeTok{animate =}\NormalTok{ T),}
\NormalTok{    tags}\OperatorTok{$}\KeywordTok{hr}\NormalTok{(),}
    \KeywordTok{numericInput}\NormalTok{(}\StringTok{"beta"}\NormalTok{, }\StringTok{"Beta value"}\NormalTok{, }\DataTypeTok{value =} \DecValTok{5}\NormalTok{),}
\NormalTok{    tags}\OperatorTok{$}\KeywordTok{hr}\NormalTok{(),}
    \KeywordTok{numericInput}\NormalTok{(}\StringTok{"alpha"}\NormalTok{, }\StringTok{"Alpha Value"}\NormalTok{, }\DataTypeTok{value =} \DecValTok{5}\NormalTok{), }\DataTypeTok{width =} \DecValTok{2}
\NormalTok{    ),}
    
    \KeywordTok{mainPanel}\NormalTok{(}
      \KeywordTok{h3}\NormalTok{(}\StringTok{"The likelihood and posterior distributions converge as the sample size n increases"}\NormalTok{),}
      \KeywordTok{br}\NormalTok{(),}
\KeywordTok{plotOutput}\NormalTok{(}\StringTok{"my_plot"}\NormalTok{, }\DataTypeTok{height =} \StringTok{"500px"}\NormalTok{)}
\NormalTok{    )}
    
\NormalTok{  )}
\NormalTok{)}

\CommentTok{# server code -------------------------------------------------------------}

\NormalTok{server <-}\StringTok{ }\ControlFlowTok{function}\NormalTok{(input, output, session)\{}
  


\CommentTok{# define posterior function -----------------------------------------------}
\NormalTok{y <-}\StringTok{ }\KeywordTok{c}\NormalTok{(}\DecValTok{24}\NormalTok{,}\DecValTok{25}\NormalTok{,}\DecValTok{31}\NormalTok{,}\DecValTok{31}\NormalTok{,}\DecValTok{22}\NormalTok{,}\DecValTok{21}\NormalTok{,}\DecValTok{26}\NormalTok{,}\DecValTok{20}\NormalTok{,}\DecValTok{16}\NormalTok{,}\DecValTok{22}\NormalTok{)}
\NormalTok{Lambda <-}\StringTok{ }\KeywordTok{mean}\NormalTok{(y)}
\NormalTok{post <-}\StringTok{ }\ControlFlowTok{function}\NormalTok{(}\DataTypeTok{n =} \DecValTok{10}\NormalTok{,}\DataTypeTok{alpha =} \DecValTok{5}\NormalTok{, }\DataTypeTok{beta =} \DecValTok{1}\OperatorTok{/}\NormalTok{Lambda)\{}

\NormalTok{lambda<-Lambda  }\CommentTok{# 1data initialization}

\NormalTok{LL <-}\StringTok{ }\KeywordTok{dpois}\NormalTok{(y,lambda)  }\CommentTok{# quartiles of a binomial}

\CommentTok{# prior}
\NormalTok{Prior <-}\StringTok{ }\KeywordTok{dgamma}\NormalTok{(y,}\DataTypeTok{shape =}\NormalTok{ alpha,}\DataTypeTok{scale =}\NormalTok{ beta)  }\CommentTok{#beta prior distribution}
  
\CommentTok{# posterior}
\NormalTok{alpha1 <-}\StringTok{ }\NormalTok{n}\OperatorTok{*}\NormalTok{lambda }\OperatorTok{+}\StringTok{ }\NormalTok{alpha}
\NormalTok{beta1 <-}\StringTok{ }\NormalTok{n}\OperatorTok{+}\StringTok{ }\NormalTok{beta}

\NormalTok{Postr <-}\StringTok{ }\KeywordTok{dgamma}\NormalTok{(y,}\DataTypeTok{shape=}\NormalTok{alpha1,}\DataTypeTok{scale =}\NormalTok{ beta1)}

\KeywordTok{ggplot}\NormalTok{(}\DataTypeTok{data =} \OtherTok{NULL}\NormalTok{, }\KeywordTok{aes}\NormalTok{(y, LL}\OperatorTok{/}\KeywordTok{max}\NormalTok{(LL), }\DataTypeTok{col =} \StringTok{"Likelihood"}\NormalTok{)) }\OperatorTok{+}\StringTok{ }\KeywordTok{geom_line}\NormalTok{(}\DataTypeTok{size =} \FloatTok{1.0}\NormalTok{)}\OperatorTok{+}
\StringTok{  }\KeywordTok{geom_line}\NormalTok{(}\KeywordTok{aes}\NormalTok{(y, Prior}\OperatorTok{/}\KeywordTok{max}\NormalTok{(Prior), }\DataTypeTok{col =} \StringTok{"Prior"}\NormalTok{), }\DataTypeTok{size =} \FloatTok{1.0}\NormalTok{) }\OperatorTok{+}\StringTok{ }
\StringTok{  }\KeywordTok{geom_line}\NormalTok{(}\KeywordTok{aes}\NormalTok{(y, Postr}\OperatorTok{/}\KeywordTok{max}\NormalTok{(Postr), }\DataTypeTok{col =} \StringTok{"Posterior"}\NormalTok{), }\DataTypeTok{size =} \FloatTok{1.0}\NormalTok{) }\OperatorTok{+}
\StringTok{  }\KeywordTok{labs}\NormalTok{(}\DataTypeTok{y =} \StringTok{"Density"}\NormalTok{, }\DataTypeTok{x=} \KeywordTok{expression}\NormalTok{(y)) }\OperatorTok{+}\StringTok{ }\KeywordTok{theme_minimal}\NormalTok{() }\OperatorTok{+}
\StringTok{  }\KeywordTok{scale_colour_manual}\NormalTok{(}\StringTok{""}\NormalTok{, }\DataTypeTok{values =} \KeywordTok{c}\NormalTok{(}\StringTok{"Likelihood"}\NormalTok{=}\StringTok{"purple"}\NormalTok{, }\StringTok{"Prior"}\NormalTok{=}\StringTok{"red"}\NormalTok{, }\StringTok{"Posterior"}\NormalTok{=}\StringTok{"blue"}\NormalTok{)) }\OperatorTok{+}
\StringTok{  }\KeywordTok{theme}\NormalTok{(}\DataTypeTok{legend.position =} \StringTok{"top"}\NormalTok{)}

\NormalTok{\}}

\NormalTok{output}\OperatorTok{$}\NormalTok{my_plot <-}\StringTok{ }\KeywordTok{renderPlot}\NormalTok{(\{}
  \KeywordTok{post}\NormalTok{(}\DataTypeTok{n=}\NormalTok{input}\OperatorTok{$}\NormalTok{n,}\DataTypeTok{alpha =}\NormalTok{ input}\OperatorTok{$}\NormalTok{alpha, }\DataTypeTok{beta =}\NormalTok{ input}\OperatorTok{$}\NormalTok{beta)}
\NormalTok{\})}
  
  
\NormalTok{\}}

\KeywordTok{shinyApp}\NormalTok{(ui, server)}
\end{Highlighting}
\end{Shaded}

\textbf{\emph{c.~Show a table of the prior, MLE and posterior estimates
of the poison mean under different choices of the gamma priors in (a)
above.}}

1.When \(\alpha=3\) and \(\beta=6\)

\begin{Shaded}
\begin{Highlighting}[]
\NormalTok{Table_Estimates <-}\StringTok{ }\ControlFlowTok{function}\NormalTok{(alpha,beta)\{}
  
\NormalTok{y <-}\StringTok{ }\KeywordTok{c}\NormalTok{(}\DecValTok{24}\NormalTok{,}\DecValTok{25}\NormalTok{,}\DecValTok{31}\NormalTok{,}\DecValTok{31}\NormalTok{,}\DecValTok{22}\NormalTok{,}\DecValTok{21}\NormalTok{,}\DecValTok{26}\NormalTok{,}\DecValTok{20}\NormalTok{,}\DecValTok{16}\NormalTok{,}\DecValTok{22}\NormalTok{)}
\NormalTok{lambda <-}\StringTok{ }\KeywordTok{mean}\NormalTok{(y)}
\NormalTok{MLE <-}\StringTok{ }\KeywordTok{dpois}\NormalTok{(y,lambda)}
\NormalTok{Prior01 <-}\StringTok{ }\KeywordTok{dgamma}\NormalTok{(y,}\DataTypeTok{shape =}\NormalTok{ alpha,}\DataTypeTok{scale =}\NormalTok{ beta)}
\NormalTok{Postr01 <-}\StringTok{ }\NormalTok{MLE}\OperatorTok{*}\NormalTok{Prior01}

\NormalTok{Estimates <-}\StringTok{ }\KeywordTok{rbind}\NormalTok{(MLE,Prior01,Postr01)}

\KeywordTok{return}\NormalTok{(Estimates)}

\NormalTok{\}}
\NormalTok{df <-}\StringTok{ }\KeywordTok{data.frame}\NormalTok{(}\KeywordTok{Table_Estimates}\NormalTok{(}\DecValTok{3}\NormalTok{,}\DecValTok{6}\NormalTok{))}
\NormalTok{df}
\end{Highlighting}
\end{Shaded}

\begin{verbatim}
##                  X1          X2           X3           X4          X5
## MLE     0.081083527 0.077191517 0.0264639843 0.0264639843 0.079016501
## Prior01 0.024420852 0.022430344 0.0126877560 0.0126877560 0.028638384
## Postr01 0.001980129 0.001731432 0.0003357686 0.0003357686 0.002262905
##                  X6          X7          X8           X9         X10
## MLE     0.073040463 0.070659927 0.064447467 0.0233562465 0.079016501
## Prior01 0.030826496 0.020536205 0.033031475 0.0411753785 0.028638384
## Postr01 0.002251582 0.001451087 0.002128795 0.0009617023 0.002262905
\end{verbatim}

\textbf{QUESTION TWO}

\textbf{\emph{a.Differentiate between Credible Intervals and the Highest
Posterior Density (HPD) in Bayesian analysis.}}

A credible interval is an interval within which an unobserved parameter
value falls with a particular probability. It is an interval in the
domain of a posterior probability distribution or a predictive
distribution.On the other hand, a highest posterior density (interval)
is basically the shortest interval on a posterior density for some given
confidence level.

\textbf{\emph{b.Graphically demonstrate this for a disease prevalence
given that out of 150 individuals 18 had the disease. Hint: assume
Binomial distribution}}

Highest Posterior Density (HPD)

\begin{Shaded}
\begin{Highlighting}[]
\KeywordTok{library}\NormalTok{(binom)}
\NormalTok{HPD <-}\StringTok{ }\KeywordTok{binom.bayes}\NormalTok{(}\DataTypeTok{x=}\DecValTok{18}\NormalTok{,}\DataTypeTok{n=}\DecValTok{150}\NormalTok{,}\DataTypeTok{type =} \StringTok{"highest"}\NormalTok{,}\DataTypeTok{conf.level =} \FloatTok{0.95}\NormalTok{,}\DataTypeTok{tol =} \FloatTok{1e-9}\NormalTok{)}
\KeywordTok{print}\NormalTok{(HPD)}
\end{Highlighting}
\end{Shaded}

\begin{verbatim}
##   method  x   n shape1 shape2      mean      lower     upper  sig
## 1  bayes 18 150   18.5  132.5 0.1225166 0.07246161 0.1754186 0.05
\end{verbatim}

\begin{Shaded}
\begin{Highlighting}[]
\KeywordTok{binom.bayes.densityplot}\NormalTok{(HPD)}
\end{Highlighting}
\end{Shaded}

\includegraphics{124384-Luycer-Bosire---Cat-I_files/figure-latex/unnamed-chunk-3-1.pdf}

Credible Interval

\begin{Shaded}
\begin{Highlighting}[]
\KeywordTok{library}\NormalTok{(binom)}
\NormalTok{Central <-}\StringTok{ }\KeywordTok{binom.bayes}\NormalTok{(}\DataTypeTok{x=}\DecValTok{18}\NormalTok{,}\DataTypeTok{n=}\DecValTok{150}\NormalTok{,}\DataTypeTok{type =} \StringTok{"central"}\NormalTok{,}\DataTypeTok{conf.level =} \FloatTok{0.95}\NormalTok{,}\DataTypeTok{tol =} \FloatTok{1e-9}\NormalTok{)}
\KeywordTok{print}\NormalTok{(Central)}
\end{Highlighting}
\end{Shaded}

\begin{verbatim}
##   method  x   n shape1 shape2      mean      lower    upper  sig
## 1  bayes 18 150   18.5  132.5 0.1225166 0.07534776 0.179135 0.05
\end{verbatim}

\begin{Shaded}
\begin{Highlighting}[]
\KeywordTok{binom.bayes.densityplot}\NormalTok{(Central)}
\end{Highlighting}
\end{Shaded}

\includegraphics{124384-Luycer-Bosire---Cat-I_files/figure-latex/unnamed-chunk-4-1.pdf}

\end{document}
