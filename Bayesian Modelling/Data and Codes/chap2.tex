% Options for packages loaded elsewhere
\PassOptionsToPackage{unicode}{hyperref}
\PassOptionsToPackage{hyphens}{url}
%
\documentclass[
]{article}
\usepackage{lmodern}
\usepackage{amssymb,amsmath}
\usepackage{ifxetex,ifluatex}
\ifnum 0\ifxetex 1\fi\ifluatex 1\fi=0 % if pdftex
  \usepackage[T1]{fontenc}
  \usepackage[utf8]{inputenc}
  \usepackage{textcomp} % provide euro and other symbols
\else % if luatex or xetex
  \usepackage{unicode-math}
  \defaultfontfeatures{Scale=MatchLowercase}
  \defaultfontfeatures[\rmfamily]{Ligatures=TeX,Scale=1}
\fi
% Use upquote if available, for straight quotes in verbatim environments
\IfFileExists{upquote.sty}{\usepackage{upquote}}{}
\IfFileExists{microtype.sty}{% use microtype if available
  \usepackage[]{microtype}
  \UseMicrotypeSet[protrusion]{basicmath} % disable protrusion for tt fonts
}{}
\makeatletter
\@ifundefined{KOMAClassName}{% if non-KOMA class
  \IfFileExists{parskip.sty}{%
    \usepackage{parskip}
  }{% else
    \setlength{\parindent}{0pt}
    \setlength{\parskip}{6pt plus 2pt minus 1pt}}
}{% if KOMA class
  \KOMAoptions{parskip=half}}
\makeatother
\usepackage{xcolor}
\IfFileExists{xurl.sty}{\usepackage{xurl}}{} % add URL line breaks if available
\IfFileExists{bookmark.sty}{\usepackage{bookmark}}{\usepackage{hyperref}}
\hypersetup{
  pdftitle={MLE Estimation},
  pdfauthor={SM Mwalili},
  hidelinks,
  pdfcreator={LaTeX via pandoc}}
\urlstyle{same} % disable monospaced font for URLs
\usepackage[margin=1in]{geometry}
\usepackage{color}
\usepackage{fancyvrb}
\newcommand{\VerbBar}{|}
\newcommand{\VERB}{\Verb[commandchars=\\\{\}]}
\DefineVerbatimEnvironment{Highlighting}{Verbatim}{commandchars=\\\{\}}
% Add ',fontsize=\small' for more characters per line
\usepackage{framed}
\definecolor{shadecolor}{RGB}{248,248,248}
\newenvironment{Shaded}{\begin{snugshade}}{\end{snugshade}}
\newcommand{\AlertTok}[1]{\textcolor[rgb]{0.94,0.16,0.16}{#1}}
\newcommand{\AnnotationTok}[1]{\textcolor[rgb]{0.56,0.35,0.01}{\textbf{\textit{#1}}}}
\newcommand{\AttributeTok}[1]{\textcolor[rgb]{0.77,0.63,0.00}{#1}}
\newcommand{\BaseNTok}[1]{\textcolor[rgb]{0.00,0.00,0.81}{#1}}
\newcommand{\BuiltInTok}[1]{#1}
\newcommand{\CharTok}[1]{\textcolor[rgb]{0.31,0.60,0.02}{#1}}
\newcommand{\CommentTok}[1]{\textcolor[rgb]{0.56,0.35,0.01}{\textit{#1}}}
\newcommand{\CommentVarTok}[1]{\textcolor[rgb]{0.56,0.35,0.01}{\textbf{\textit{#1}}}}
\newcommand{\ConstantTok}[1]{\textcolor[rgb]{0.00,0.00,0.00}{#1}}
\newcommand{\ControlFlowTok}[1]{\textcolor[rgb]{0.13,0.29,0.53}{\textbf{#1}}}
\newcommand{\DataTypeTok}[1]{\textcolor[rgb]{0.13,0.29,0.53}{#1}}
\newcommand{\DecValTok}[1]{\textcolor[rgb]{0.00,0.00,0.81}{#1}}
\newcommand{\DocumentationTok}[1]{\textcolor[rgb]{0.56,0.35,0.01}{\textbf{\textit{#1}}}}
\newcommand{\ErrorTok}[1]{\textcolor[rgb]{0.64,0.00,0.00}{\textbf{#1}}}
\newcommand{\ExtensionTok}[1]{#1}
\newcommand{\FloatTok}[1]{\textcolor[rgb]{0.00,0.00,0.81}{#1}}
\newcommand{\FunctionTok}[1]{\textcolor[rgb]{0.00,0.00,0.00}{#1}}
\newcommand{\ImportTok}[1]{#1}
\newcommand{\InformationTok}[1]{\textcolor[rgb]{0.56,0.35,0.01}{\textbf{\textit{#1}}}}
\newcommand{\KeywordTok}[1]{\textcolor[rgb]{0.13,0.29,0.53}{\textbf{#1}}}
\newcommand{\NormalTok}[1]{#1}
\newcommand{\OperatorTok}[1]{\textcolor[rgb]{0.81,0.36,0.00}{\textbf{#1}}}
\newcommand{\OtherTok}[1]{\textcolor[rgb]{0.56,0.35,0.01}{#1}}
\newcommand{\PreprocessorTok}[1]{\textcolor[rgb]{0.56,0.35,0.01}{\textit{#1}}}
\newcommand{\RegionMarkerTok}[1]{#1}
\newcommand{\SpecialCharTok}[1]{\textcolor[rgb]{0.00,0.00,0.00}{#1}}
\newcommand{\SpecialStringTok}[1]{\textcolor[rgb]{0.31,0.60,0.02}{#1}}
\newcommand{\StringTok}[1]{\textcolor[rgb]{0.31,0.60,0.02}{#1}}
\newcommand{\VariableTok}[1]{\textcolor[rgb]{0.00,0.00,0.00}{#1}}
\newcommand{\VerbatimStringTok}[1]{\textcolor[rgb]{0.31,0.60,0.02}{#1}}
\newcommand{\WarningTok}[1]{\textcolor[rgb]{0.56,0.35,0.01}{\textbf{\textit{#1}}}}
\usepackage{graphicx,grffile}
\makeatletter
\def\maxwidth{\ifdim\Gin@nat@width>\linewidth\linewidth\else\Gin@nat@width\fi}
\def\maxheight{\ifdim\Gin@nat@height>\textheight\textheight\else\Gin@nat@height\fi}
\makeatother
% Scale images if necessary, so that they will not overflow the page
% margins by default, and it is still possible to overwrite the defaults
% using explicit options in \includegraphics[width, height, ...]{}
\setkeys{Gin}{width=\maxwidth,height=\maxheight,keepaspectratio}
% Set default figure placement to htbp
\makeatletter
\def\fps@figure{htbp}
\makeatother
\setlength{\emergencystretch}{3em} % prevent overfull lines
\providecommand{\tightlist}{%
  \setlength{\itemsep}{0pt}\setlength{\parskip}{0pt}}
\setcounter{secnumdepth}{-\maxdimen} % remove section numbering

\title{MLE Estimation}
\author{SM Mwalili}
\date{1/21/2021}

\begin{document}
\maketitle

\# Overview

This vignette introduces the idea of ``conjugate prior'' distributions
for Bayesian inference for a continuous parameter. You should be
familiar with \href{bayes_beta_binomial.html}{Bayesian inference for a
binomial proportion}.

\hypertarget{conjugate-priors-for-binomial-proportion}{%
\section{Conjugate Priors for binomial
proportion}\label{conjugate-priors-for-binomial-proportion}}

\hypertarget{background}{%
\subsection{Background}\label{background}}

In \href{bayes_beta_binomial.html}{this example} we considered the
following problem.

Suppose we sample 100 elephants from a population, and measure their DNA
at a location in their genome (``locus'') where there are two types
(``alleles''), which it is convenient to label \(a\) and \(b\).

In my sample, I observe that 30 of the elephants have the ``\(a\)''
allele and 70 have the ``\(b\)'' allele. What can I say about the
frequency, \(\theta\), of the ``\(a\)'' allele in the population?

The example showed how to compute the posterior distribution for
\(\theta\), using a \emph{uniform prior distribution}. We saw that,
conveniently, the posterior distribution for \(\theta\) is a Beta
distribution.

Here we generalize this calculation to the case where the prior
distribution on \(\theta\) is a Beta distribution. We will find that, in
this case, the posterior distribution on \(\theta\) is again a Beta
distribution. The property where the posterior distribution comes from
the same family as the prior distribution is very convenient, and so has
a special name: it is called ``conjugacy''. We say ``The Beta
distribution is the conjugate prior distribution for the binomial
proportion''.

\hypertarget{details}{%
\subsection{Details}\label{details}}

As before we use Bayes Theorem which we can write in words as
\[\text{posterior} \propto 
\text{likelihood} \times \text{prior},\] or in mathematical notation as
\[ p(\theta | D) \propto p(D | \theta) p(\theta),\] where \(D\) denotes
the observed data.

In this case, the likelihood \(p(D | \theta)\) is given by
\[p(D | \theta) \propto \theta^{30} (1-\theta)^{70}\]

If our prior distribution on \(\theta\) is a Beta distribution, say
Beta\((\alpha,\beta)\), then the prior density \(p(\theta)\) is
\[p(\theta) \propto \theta^{\alpha-1}(1-\theta)^{\beta-1} \qquad (\theta \in [0,1]).\]

Combining these two we get:
\[p(\theta | D) \propto \theta^{30} (1-\theta)^{70} \theta^{\alpha-1} (1-\theta)^{\beta-1}\\
\propto \theta^{30+\alpha-1}(1-\theta)^{70+\beta-1}\]

At this point we again apply the ``trick'' of recognizing this density
as the density of a Beta distribution - specifically, the Beta
distribution with parameters \((30+\alpha,70+\beta)\).

\hypertarget{generalization}{%
\subsection{Generalization}\label{generalization}}

Of course, there is nothing special about the 30 ``\(a\)'' alleles and
70 ``\(b\)'' alleles we observed here. Suppose we observed \(x\) of the
``\(a\)'' allele and \(n-x\) of the ``\(b\)'' allele. Then the
likelihood becomes
\[p(D | \theta) \propto \theta^{x} (1-\theta)^{n-x},\] and you should be
able to show (Exercise) that the posterior is
\[\theta|D \sim \text{Beta}(x+\alpha, n-x+\beta).\]

\hypertarget{r-coding-for-the-binomail-example}{%
\subsection{R coding for the Binomail
Example}\label{r-coding-for-the-binomail-example}}

\hypertarget{likelihood}{%
\subsubsection{Likelihood}\label{likelihood}}

\begin{Shaded}
\begin{Highlighting}[]
\CommentTok{# Plot likelihood and log-likelihood of binomial example}

\CommentTok{# Specify first a grid of theta values upon which the likelihood will be determined}

\NormalTok{theta <-}\StringTok{ }\KeywordTok{seq}\NormalTok{(}\FloatTok{0.001}\NormalTok{,}\FloatTok{0.999}\NormalTok{,}\FloatTok{0.001}\NormalTok{)}

\CommentTok{# There are 30 successes out of 100 experiments}
\NormalTok{x <-}\StringTok{ }\DecValTok{30}
\NormalTok{n <-}\StringTok{ }\DecValTok{100}

\CommentTok{# Explicit computation of likelihood and log-likelihood}

\NormalTok{binom <-}\StringTok{ }\KeywordTok{choose}\NormalTok{(n,x)}
\NormalTok{lik <-}\StringTok{ }\NormalTok{binom}\OperatorTok{*}\NormalTok{theta}\OperatorTok{**}\NormalTok{x}\OperatorTok{*}\NormalTok{(}\DecValTok{1}\OperatorTok{-}\NormalTok{theta)}\OperatorTok{**}\NormalTok{(n}\OperatorTok{-}\NormalTok{x)}
\NormalTok{llik <-}\StringTok{ }\KeywordTok{log}\NormalTok{(lik)}

\CommentTok{# Determine MLE and (log)-likelihood value at MLE}

\NormalTok{mle <-}\StringTok{ }\NormalTok{x}\OperatorTok{/}\NormalTok{n}
\KeywordTok{print}\NormalTok{(mle)}
\end{Highlighting}
\end{Shaded}

\begin{verbatim}
## [1] 0.3
\end{verbatim}

\begin{Shaded}
\begin{Highlighting}[]
\NormalTok{likmle <-}\StringTok{ }\NormalTok{binom}\OperatorTok{*}\NormalTok{mle}\OperatorTok{**}\NormalTok{x}\OperatorTok{*}\NormalTok{(}\DecValTok{1}\OperatorTok{-}\NormalTok{mle)}\OperatorTok{**}\NormalTok{(n}\OperatorTok{-}\NormalTok{x)}
\NormalTok{llikmle <-}\StringTok{ }\KeywordTok{log}\NormalTok{(likmle)}

\CommentTok{# Plot likelihood and log-likelihood}
\CommentTok{# par statements imply 2 square figures in one graph (2 rows, 1 column)}

\KeywordTok{par}\NormalTok{(}\DataTypeTok{mfrow=}\KeywordTok{c}\NormalTok{(}\DecValTok{1}\NormalTok{,}\DecValTok{2}\NormalTok{), }\DataTypeTok{pty=}\StringTok{"s"}\NormalTok{)}

\CommentTok{# Plot statement}

\KeywordTok{plot}\NormalTok{(theta,lik,}\DataTypeTok{xlab=}\KeywordTok{expression}\NormalTok{(theta) ,}\DataTypeTok{ylab=}\StringTok{"Likelihood"}\NormalTok{,}\DataTypeTok{type=}\StringTok{"n"}\NormalTok{,}\DataTypeTok{main=}\StringTok{""}\NormalTok{,}
\DataTypeTok{bty=}\StringTok{"l"}\NormalTok{,}\DataTypeTok{cex.lab=}\FloatTok{1.5}\NormalTok{,}\DataTypeTok{cex=}\FloatTok{1.2}\NormalTok{,}\DataTypeTok{cex.axis=}\FloatTok{1.3}\NormalTok{)}
\KeywordTok{lines}\NormalTok{(theta,lik,}\DataTypeTok{col=}\StringTok{"red"}\NormalTok{,}\DataTypeTok{lwd=}\DecValTok{3}\NormalTok{)}
\KeywordTok{arrows}\NormalTok{(mle,likmle,mle,}\DecValTok{0}\NormalTok{,}\DataTypeTok{col=}\StringTok{"red"}\NormalTok{)}
\KeywordTok{text}\NormalTok{(}\FloatTok{0.1}\NormalTok{,}\FloatTok{0.25}\NormalTok{,}\StringTok{"(a)"}\NormalTok{,}\DataTypeTok{cex=}\FloatTok{1.4}\NormalTok{)}
\KeywordTok{plot}\NormalTok{(theta,llik,}\DataTypeTok{xlab=}\KeywordTok{expression}\NormalTok{(theta),}\DataTypeTok{ylab=}\StringTok{"Log-likelihood"}\NormalTok{,}\DataTypeTok{type=}\StringTok{"n"}\NormalTok{,}
\DataTypeTok{main=}\StringTok{""}\NormalTok{,}\DataTypeTok{bty=}\StringTok{"l"}\NormalTok{,}\DataTypeTok{cex.lab=}\FloatTok{1.5}\NormalTok{,}\DataTypeTok{cex=}\FloatTok{1.2}\NormalTok{,}\DataTypeTok{cex.axis=}\FloatTok{1.3}\NormalTok{)}
\KeywordTok{text}\NormalTok{(}\FloatTok{0.1}\NormalTok{,}\OperatorTok{-}\DecValTok{3}\NormalTok{,}\StringTok{"(b)"}\NormalTok{,}\DataTypeTok{cex=}\FloatTok{1.4}\NormalTok{)}
\KeywordTok{lines}\NormalTok{(theta,llik,}\DataTypeTok{col=}\StringTok{"blue"}\NormalTok{,}\DataTypeTok{lwd=}\DecValTok{3}\NormalTok{)}
\KeywordTok{arrows}\NormalTok{(mle,llikmle,mle,}\KeywordTok{min}\NormalTok{(llik),}\DataTypeTok{col=}\StringTok{"blue"}\NormalTok{)}
\end{Highlighting}
\end{Shaded}

\includegraphics{chap2_files/figure-latex/unnamed-chunk-1-1.pdf}

\hypertarget{combination-prior-likelihood-for-binomial-example}{%
\subsection{Combination prior \& likelihood for binomial
example}\label{combination-prior-likelihood-for-binomial-example}}

\begin{Shaded}
\begin{Highlighting}[]
\CommentTok{# Now only 1 figure}
\CommentTok{# Size figure is maximized}

\KeywordTok{par}\NormalTok{(}\DataTypeTok{mfrow=}\KeywordTok{c}\NormalTok{(}\DecValTok{1}\NormalTok{,}\DecValTok{1}\NormalTok{))}
\KeywordTok{par}\NormalTok{(}\DataTypeTok{pty=}\StringTok{"m"}\NormalTok{)}

\CommentTok{# Grid of theta values}

\NormalTok{n <-}\StringTok{ }\DecValTok{100}
\NormalTok{x <-}\StringTok{ }\DecValTok{30}
\NormalTok{theta <-}\StringTok{ }\KeywordTok{seq}\NormalTok{(}\DecValTok{0}\NormalTok{,}\FloatTok{1.}\NormalTok{,}\FloatTok{0.001}\NormalTok{)}

\CommentTok{# Prior based on ECASS 2 data}
\CommentTok{# Historical binomial likelihood turned into a beta density}

\NormalTok{alpha <-}\DecValTok{10}
\NormalTok{beta <-}\StringTok{ }\DecValTok{1}
\CommentTok{# Parameters of beta density}

\NormalTok{betatheta <-}\StringTok{ }\KeywordTok{dbeta}\NormalTok{(theta,alpha,beta)}

\CommentTok{# expression(theta) produces a Greek symbol}

\KeywordTok{plot}\NormalTok{(theta, betatheta,}\DataTypeTok{xlab=}\KeywordTok{expression}\NormalTok{(theta),}\DataTypeTok{ylab=}\StringTok{""}\NormalTok{,}\DataTypeTok{type=}\StringTok{"n"}\NormalTok{,}\DataTypeTok{ylim=}\KeywordTok{c}\NormalTok{(}\DecValTok{0}\NormalTok{,}\DecValTok{18}\NormalTok{),}
     \DataTypeTok{bty=}\StringTok{"l"}\NormalTok{,}\DataTypeTok{cex.lab=}\FloatTok{1.8}\NormalTok{)}
\KeywordTok{lines}\NormalTok{(theta,betatheta,}\DataTypeTok{lwd=}\DecValTok{2}\NormalTok{,}\DataTypeTok{col=} \StringTok{"red2"}\NormalTok{)}

\CommentTok{# Likelihood of ECASS 3 study}

\NormalTok{p <-}\StringTok{ }\NormalTok{x }
\NormalTok{q <-}\StringTok{ }\NormalTok{n }\OperatorTok{-}\StringTok{ }\NormalTok{x}

\NormalTok{lik <-}\StringTok{ }\KeywordTok{dbeta}\NormalTok{(theta,p,q)}
\KeywordTok{lines}\NormalTok{(theta,lik,}\DataTypeTok{lwd=}\DecValTok{2}\NormalTok{,}\DataTypeTok{col=}\StringTok{"green"}\NormalTok{)}

\CommentTok{# Posterior combining ECASS 2 prior with ECASS 3 likelihood}
\NormalTok{alpha2 <-}\StringTok{ }\NormalTok{alpha}\OperatorTok{+}\NormalTok{x}
\NormalTok{beta2 <-}\StringTok{ }\NormalTok{beta }\OperatorTok{+}\NormalTok{n}\OperatorTok{-}\NormalTok{x}
\NormalTok{betatheta <-}\StringTok{ }\KeywordTok{dbeta}\NormalTok{(theta,alpha2,beta2)}
\KeywordTok{lines}\NormalTok{(theta, betatheta,}\DataTypeTok{lwd=}\DecValTok{2}\NormalTok{,}\DataTypeTok{col=}\StringTok{"blue"}\NormalTok{)}
\KeywordTok{legend}\NormalTok{(}\DataTypeTok{x=}\StringTok{"topright"}\NormalTok{,}\KeywordTok{c}\NormalTok{(}\StringTok{"Prior"}\NormalTok{,}\StringTok{"Likelihood"}\NormalTok{,}\StringTok{"Posterior"}\NormalTok{),}\DataTypeTok{lty=}\DecValTok{1}\NormalTok{,}\DataTypeTok{col=}\KeywordTok{c}\NormalTok{(}\StringTok{"red2"}\NormalTok{,}\StringTok{"green"}\NormalTok{,}\StringTok{"blue"}\NormalTok{),}\DataTypeTok{bty=}\StringTok{"n"}\NormalTok{,}\DataTypeTok{lwd=}\DecValTok{2}\NormalTok{)}
\end{Highlighting}
\end{Shaded}

\includegraphics{chap2_files/figure-latex/unnamed-chunk-2-1.pdf}

\hypertarget{rightarrowtry-it-yourself}{%
\paragraph{\texorpdfstring{\emph{\(\rightarrow\)Try it
yourself}}{\textbackslash rightarrowTry it yourself}}\label{rightarrowtry-it-yourself}}

In the cell below, write a code which calls the
\texttt{Binomial\_post(alpha,beta)} for any choice of alpha and beta.

\begin{Shaded}
\begin{Highlighting}[]
\CommentTok{#function}
\CommentTok{#}
\end{Highlighting}
\end{Shaded}

\hypertarget{rightarrowtry-it-yourself-1}{%
\paragraph{\texorpdfstring{\emph{\(\rightarrow\)Try it
yourself}}{\textbackslash rightarrowTry it yourself}}\label{rightarrowtry-it-yourself-1}}

In the cell below, write a function implementing the
\texttt{Binomial\_post(alpha,beta)} function looping over alpha and beta
values. Plot the results on the same axis.

\begin{Shaded}
\begin{Highlighting}[]
\CommentTok{#}

\CommentTok{#}
\end{Highlighting}
\end{Shaded}

\hypertarget{summary}{%
\subsection{Summary}\label{summary}}

When doing Bayesian inference for a binomial proportion, \(\theta\), if
the prior distribution is a Beta distribution then the posterior
distribution is also Beta.

We say ``the Beta distribution is the conjugate prior for a binomial
proportion''.

\hypertarget{exercise}{%
\section{Exercise}\label{exercise}}

Show that the Gamma distribution is the conjugate prior for a Poisson
mean.

That is, suppose we have observations \(X\) that are Poisson
distributed, \(X \sim Poi(\mu)\). Assume that your prior distribution on
\(\mu\) is a Gamma distribution with parameters \(n\) and \(\lambda\).
Show that the posterior distribution on \(\mu\) is also a Gamma
distribution.

Hint: you should take the following steps. 1. write down the likelihood
\(p(X|\mu)\) for \(\mu\) (look up the Poisson distribution if you cannot
remember it). 2. Write down the prior density for \(\mu\) (look up the
density of a Gamma distribution if you cannot remember it). 3. Multiply
them together to obtain the posterior density (up to a constant of
proportionality), and notice that it has the same form as the gamma
distribution.

\end{document}
